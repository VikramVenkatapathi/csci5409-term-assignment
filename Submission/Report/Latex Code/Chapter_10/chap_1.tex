
\newpage

% \begin{flushright}
%     \vspace{10cm}
%     \rule{18cm}{5pt}
%     \rule{18cm}{2pt}\vskip1cm
%     \begin{center}
%     \begin{bfseries}
%         \Huge{Explore \& Build a Use Case}\\
%     \end{bfseries}
%     \end{center}
%     \vspace{1cm}
%     \rule{18cm}{2pt}
%     \rule{18cm}{5pt}
% \end{flushright}
\clearpage

\chapter{Application Evolution and Future Features}
If I were to continue the development of my application, I would make the following changes:
\begin{enumerate}
    \item \textbf{User Authentication and Authorization}: Implement a robust user authentication and authorization system to secure user data and control access to various functionalities. Use AWS Cognito for user management, enabling secure sign-up, login, and user attribute management.
    
    \item \textbf{User Profile Management}: Allow users to update their profiles, including profile pictures and other relevant information. Use AWS Lambda to handle profile updates and store user data in DynamoDB.

    \item \textbf{Serverless File Uploads}: Enhance the file upload feature to support multiple file types and larger files. Utilize AWS API Gateway and Lambda to process and store user-uploaded files securely in Amazon S3.
    
    \item \textbf{Real-time Notifications}: Integrate WebSocket functionality using AWS API Gateway and AWS Lambda to provide real-time notifications to users, such as when image processing is completed or when new reports are available.
    
    \item \textbf{Application Monitoring and Analytics}: Implement monitoring and analytics using AWS CloudWatch and AWS X-Ray to gain insights into application performance, identify bottlenecks, and optimize resource utilization.
    
    \item User\textbf{ Access Control for Reports}: Enhance security by allowing users to access only their own reports. Implement AWS IAM-based access control to restrict access to specific S3 objects based on user identity.
    
    \item \textbf{Data Backup and Archiving}: Set up automated data backup and archiving processes using AWS Data Lifecycle Manager and Glacier for long-term storage of historical data and reports.
    
    \item \textbf{Performance Optimization}: Continuously optimize Lambda functions, frontend code, and database queries to improve application responsiveness and reduce latency.
    
    \item \textbf{Cost Optimization}: Regularly review cost metrics and optimize resource allocation. Utilize AWS Cost Explorer to analyze costs and identify opportunities for cost-saving measures.
    
    \item \textbf{Geolocation Services}: Integrate geolocation services like Amazon Location Service to add location-based functionality, such as tagging images with location data or displaying image locations on a map.
    
    \item \textbf{Enhanced Image Analysis}: Explore additional image analysis capabilities using AWS services like Amazon Rekognition Custom Labels to create custom image classification models tailored to specific use cases.
    
    \item \textbf{Data Privacy and Compliance}: Implement data privacy and compliance measures to adhere to relevant regulations. Utilize AWS Key Management Service (KMS) for data encryption at rest and in transit.
\end{enumerate}

By incorporating these features and leveraging various AWS cloud mechanisms, the application can evolve into a comprehensive and powerful image analysis platform. Continuous development and improvement will ensure that the application meets user needs, maintains high performance, and remains cost-efficient in the dynamic cloud environment.




\newpage

\comment{
\chapter*{Revision History}

\begin{center}
    \begin{tabular}{|c|c|c|c|}
        \hline
	    Date & Version & Description & Author\\
        \hline
	     04-Mar-2021 & 1.0 & Interaction Diagram Document - Initial Release. & All\\
        \hline
	    %31 & 32 & 33 & 34\\
        % \hline
    \end{tabular}
\end{center}




\newpage
\tableofcontents
}

\comment{
\chapter{Interaction Diagram}
\begin{figure}[htp]
    \centering
    \includegraphics[width=17.5cm]{04 - Interaction Diagram/Quiz Application-2.png}
    \caption{\textbf{\textit{Login functionality - Interaction diagram}}}
    \label{fig:my_label}
\end{figure}

\begin{figure}[htp]
    \centering
    \includegraphics[width=17.5cm]{04 - Interaction Diagram/Quiz Application-1.png}
    \caption{\textbf{\textit{Quiz Application - Interaction diagram }}}
    \label{fig:my_label}
\end{figure}
}